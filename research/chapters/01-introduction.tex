
% Puts a line after the Chapter Heading
\vspace{-1cm}\noindent\rule{\textwidth}{0.4pt}
\section{Background of the Study}
% Introductory Paragraph
% Background to Agriculture

Agriculture plays a major role in feeding billions of people all
across the globe. It has also provided a means for people
to grow revenue and sustain their lives. In fact according to the 
World Bank Organization, "Agricultural development 
is one of the most powerful tools to end extreme poverty, 
boost shared prosperity and feed a projected 9.7 billion people by 2050.
Agriculture plays an integral 
part of the Philippine Economy, 
in fact according to the Philippine Statistics Authority, the Gross 
Value Added (GVA) in agriculture of the year 2019
is valued at 1.78 Trillion Pesos. 
It also contributes to the diet of Filipinos producing various 
ingredients of that are necessary for the Filipino Cuisine such as rice 
being the staple ingredient and other leafy vegetables such as cabbage,
saluyot and ampalaya. \\


% Further discussions introducing the Problem/Issues
% Intoduce problems 

Each year, farmers spend a lot of money on disease management, they 
often do so without proper technical support leading poor disease 
control, pollution and harmful result. Plant diseases cause 
substantial loss to farmers resulting to large economic loss.
That's why accurate identification and diagnosis of plant diseases is 
essential; especially now in the era of globalization and climate 
change. \\

During the past few years, there has been a rapid advancement to the
area of Machine Learning. Because of this there are now self-driving cars 
that can fully control a vehicle with minimal efforts required from the
driver. Engineers and researchers have achieved this through the use 
of different types of Neural Networks. \\

Convolutional Neural Network is a type of Neural Network that specializes 
in finding patterns to images. It is being used in almost all areas of science, 
for instance (Hadush et al.) has used CNNs to detect breast 
cancer from mammogram(MG) images achieving a detection accuracy 91.86\%, 
\\

% https://arxiv.org/abs/2003.07911

This study aims to develop a mobile application that uses a 
CNN Model  to detect and classify potential diseases in 
plants. Due to the current pandemic, the researcher decided to use publicly 
available image datasets — which will be used to train several CNN Models 
that would be embedded in the application — instead of making one. The researcher 
found publicly available image datasets of the following plants:
apple, cherry, citrus, corn, grape, peach, pepper, potato,
rice, strawberry (this selection is tentative). These images of different types 
of plants will be used to train the CNN Model and be able to detect diseases from 
these plants using an image of the plant leaves. \\ 

Because the model inference will
happen on the device itself without relying on a cloud-based server, 
farmers and other people would be able to input an image to this application 
and use this app without having to rely on an 
existing data or internet connection. 


% With this information, the researcher will create an application
% that allow farmers to input images of plants which will then be 
% passed to the pretrained CNN Model that will classify the disease 
% of the given image.. This will happen without requiring the user to 
% have an internet connection which would be helpful to guide farmers 
% that are living in a place without a data connection.

\section{Conceptual Framework}

\section{Theoretical Framework}

\section{Statement of the Problem}

    \begin{enumerate}
    \item How effective is the model in predicting the
    type of disease present in the given image? 
    In terms of:

        \begin{enumerate}
            \item F1 Score 
            \item Speed of Classification 
        \end{enumerate}

    \item How comparable is the model made by the researcher 
    to other methods of detecting plant disease? 
    \end{enumerate}

\section{Hypotheses}
    \begin{enumerate}
        \item The F1 Score and the Speed of Classification implies that
              the model is not effective in predicting the tyoe of disease 
              present in the given image.
        \item The model is not comparable to other methods of detecting plant
              disease.
    \end{enumerate}

\section{Importance of the Study}
The significance of the study lies on the
fact that plant disease diagnosis plays a huge role 
in minimizing the monetary and material loss in
agriculture caused by various plant diseases.  
That’s why proper plant disease diagnosis is a
detrimental process in securing the value of 
crops grown. The study proposes an application 
that can be used to scan images of plant leaves 
which would then be used to detect various plant 
diseases which may be present in an image, this 
application is designed to run the disease diagnosis 
without the need of internet or data connection.
This study would certainly benefit farmers who 
have no access to internet connection to diagnose
their plants and prevent further crop losses. 


\section{Definition of Terms}
    \begin{itemize}
        \item Model \\
              - A machine learning model is a file that has
              been trained to recognize certain types of patterns.
              You train a model over a set of data, providing it an
              algorithm that it can use to reason over and learn 
              from those data.

        \item Convolutional Neural Networks \\
              - A Convolutional neural network (CNN) is a neural network 
              that has one or more convolutional layers and are used mainly
              for image processing, classification, segmentation and also 
              for other auto correlated data.

        \item Gross Value Added \\
              - Gross value added is the measure of the value of 
              goods and services produced in an area, industry or sector 
              of an economy.

        \item F1 Score \\
              - The F-score, also called the F1-score, is a measure of a model's accuracy 
              on a dataset. A good F1 score means that you have low false positives and 
              low false negatives, so you're correctly identifying real threats and you are not disturbed by false alarms
              
        \item Machine Learning \\
              - Machine learning is an application of artificial intelligence 
              (AI) that provides systems the ability to automatically learn and
              improve from experience without being explicitly programmed. 
              Machine learning focuses on the development of computer programs 
              that can access data and use it to learn for themselves.

        \item Classification \\
              - classification refers to a predictive modeling problem where a
              class label is predicted for a given example of input data.
              
        \item Inference \\
              -  Inference refers to the process of using a trained machine learning algorithm to make a prediction
    \end{itemize}


